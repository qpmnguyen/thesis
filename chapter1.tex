\chapter{Introduction}

\section{The human gut microbiome and population health}
\subsection{Overview of the human gut microbiome}

The human microbiome is the collection microorganisims (which includes bacteria, protozoa, archaea, fungi, viruses, and their genes) that participate in a symbiotic co-existence with their hosts \cite{ursell2012defining}. It is difficult to study human microbiomes outside the host due to difficulties in culturing the majority of organisms \cite{}, however advances in sequencing technologies have allowed researchers to glimpse into the inner workings of these complex communities \cite{}. Most notably, different body sites harbor unique environmental determinants that give rise to distinct groups of microbes \cite{consortium2012structure}. For example, in the oral microbiome, oral surfaces have different surface receptors \cite{gibbons1989bacterial}, promoting only microbes with specific adhesins that are complementary \cite{aas2005defining}. This resulted in differences such as \emph{Streptococcus mitis} bv.2 species being represted in the tongue dorsum but not even detected in the related lateral tongue surface \cite{aas2005defining}. Even though the degree of diversity across body sites extend across the entire tree of life, most studies so-far have been focused on profiling bacteria, which exists in high-abundance and is relatively easy to profile \cite{}. 

The gut environment also promotes a very specific group of microbes which varies across the digestive track \cite{mailhe2018repertoire, donaldson2016gut}. Most microbiome research has been focused on the colon via fecal samples \cite{tang2020current}, where it has been estimated to contain the highest microbial density recorded in any habitat \cite{}. The gut microbiome is acquired at birth \cite{} via maternal transfer \cite{}. The gut community matures over time, increasing in diversity and reacing an adult-like state is achieved around 2-3 years of age, where it is characterized mainly by members of the Firmicutes, Actinobacteria, and Bacteroidetes phyla, with \emph{Bacteroides}, \emph{Faecalibacterium}, and \emph{Bifidobacterium} as the most abundant genera \cite{king2019baseline, metahitconsortiumadditionalmembers2011enterotypes}. Many of the species identified to be in the gut microbiome cannot be found in other habitats, suggesting a strong co-evolutionary relationship with human hosts \cite{ley2006ecological}.  

Various environmental factors can shape the composition of the gut microbiome. In early life, the mode of delivery and breastfeeding status are significant modifiers of composition. Infants who were born via vaginal delivery has increased abundances of \emph{Bacteroides}, \emph{Pectobacterium}, and \emph{Bifidobacterium} genera, while those born via Cesarian section have decreased diversity and higher propensity to be colonized by \emph{Staphylococcus} and members of the \emph{Clostridum} cluster \cite{madan2016effects, kim2020delayed, stewart2018temporal}. Breastfeeding is assocated with lower levels of \emph{Escherichia coli}, \emph{Tyzzerella nexilis}, and \emph{Roseburia intestinalis} while on the other hand promoting the coloization of various \emph{Bifidobacterium} species such as \emph{B. breve}, \emph{B. dentium} which harbors specific genes that help in digesting complex oligosaccharides \cite{stewart2018temporal, vatanen2018human}. Among adults, diet is of particular interest. Studies have shown that the Western diet, high in saturated and trans fats while low in mono and polyunsaturated fats, is associated with decreased abundance in \emph{Bifidobacterium}, \emph{Eubacterium}, and \emph{Lactobacillus} genera \cite{wu2011linking}. Other extrinsic and intrinsic factors such as smoking status \cite{biedermann2013smoking}, alcohol consumption \cite{dubinkina2017links}, etc. also contribute to microbiome modulation (reviewed in \cite{schmidt2018humana}). In conclusion, the gut microbiome is highly malleable yet intrinsic part of human existence. 

\subsection{Outcomes associated with changes in gut microbiome}

Observed shifts in microbiome composition are often associated with health outcomes. As such, there is a great interest epidemiological applications, where the microbiome can be used as a marker or identified as a causal agent in human disease \cite{foxman2015use}. 

Studies have linked the microbiome to obesity, inflammatory bowel disease, colorectal cancer. 





Microbiomes associated with diseased individuals are often classified as being in a state of "dysbiosis" \cite{duvallet2017metaanalysis}. However, the definition of the term is vague and often inconsistent, leading to  


\subsection{Challenges in determining microbial biomarkers}
This section is about the differences and variability between and within the same person with regards to their microbiome. Here I attempt to highlight some changes in the overall interpreting results from differential abundance analyses. (from a causal point of view)


\section{Approaching microbiome research from a mechanistic perspective}

\subsection{Characterizing microbial function}
In this section, we review technologies that are available to characterize microbiome function. 
\subsubsection{Metagenomics}
\subsubsection{Metatranscriptomics}
\subsubsection{Proteomics}
\subsubsection{Metabolomics}

\subsection{Gut microbial function}
A section what it means to talk about microbiome functions and how do those technologies talk about the different ways microbes interact with the host. 


\section{An integrative approach to incorporates both structure and function}

\subsection{Importance of taxa-function relationships}
Here we talk about the definition of the taxa-function relationship and how they are important. The key idea here is that therapeutics and environmental factors can influence taxa but then those taxa interact with other members of the community to form functional components that interact with the host. 


\subsection{Biodiversity, ecosystem functioning, and community resilience}
Taxa-function relationships are part of an ecology concept titled "biodiversity and ecosystem functioning" which relates to community resilience and probably long-term health outcomes. 


\section{Summary}
