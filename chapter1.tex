\chapter{Introduction}

\section{The human gut microbiome and population health}
\subsection{Overview of the human gut microbiome}

The human microbiome is the collection microorganisims (which includes bacteria, protozoa, archaea, fungi, viruses, and their genes) that participate in a symbiotic co-existence with their hosts \cite{ursell2012defining}.  Different body sites harbor unique environmental factors that give rise to distinct microbial communities \cite{consortium2012structure}. For example, in the oral microbiome, oral surfaces have different surface receptors \cite{gibbons1989bacterial}, promoting only microbes with specific adhesins that are complementary \cite{aas2005defining}. This resulted in differences such as \emph{Streptococcus mitis} bv. 2 being represted in the tongue dorsum but not even detected in the lateral tongue surface \cite{aas2005defining}. 

Similarly, the gut microbiome 


are dominated by \emph{Streptococcus} 

\subsection{Outcomes associated with changes in gut microbiome}
How the microbiome is related to health and a review of some important health outcomes for adults and infants that relate to the microbiome


\subsection{Challenges in determining microbial biomarkers}
This section is about the differences and variability between and within the same person with regards to their microbiome. Here I attempt to highlight some changes in the overall interpreting results from differential abundance analyses. (from a causal point of view)


\section{Approaching microbiome research from a mechanistic perspective}

\subsection{Characterizing microbial function}
In this section, we review technologies that are available to characterize microbiome function. 
\subsubsection{Metagenomics}
\subsubsection{Metatranscriptomics}
\subsubsection{Proteomics}
\subsubsection{Metabolomics}

\subsection{Host-microbial interactions}
A section what it means to talk about microbiome functions and how do those technologies talk about the different ways microbes interact with the host. 


\section{An integrative approach to incorporates both structure and function}

\subsection{Importance of taxa-function relationships}
Here we talk about the definition of the taxa-function relationship and how they are important. The key idea here is that therapeutics and environmental factors can influence taxa but then those taxa interact with other members of the community to form functional components that interact with the host. 


\subsection{Biodiversity, ecosystem functioning, and community resilience}
Taxa-function relationships are part of an ecology concept titled "biodiversity and ecosystem functioning" which relates to community resilience and probably long-term health outcomes. 


\section{Summary}
