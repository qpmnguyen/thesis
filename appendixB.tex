\chapter{Supporting material for ``Associations between the gut microbiome and metabolime in early life"}

\section{List of abbreviations}
NHBCS: New Hampshire Birth Cohort Study
NMR: Nuclear Magnetic Resonance
PCoA: Principal Coordinates Analysis
gUniFrac: Generalized Unique Fraction 
ASV: Amplicon Sequence Variant
FDR: False Discovery Rate
sCCA/CCA: Sparse Canonical Correlation Analysis
SCFA: Short Chain Fatty Acids
RF: Random Forest
EN: Elastic Net
SVM-RBF: Support Vector Machines with Radial Basis kernel Function
SPLS: Sparse Partial Least Squares
CLR:  Centered Log Ratio transformation

\section{Availability of data and materials}

The 16S rRNA gene sequencing datasets used in this study are stored in the National Center for Biotechnology Information (NCBI) Sequence Read Archive: \url{http://www.ncbi.nlm.nih.gov/sra} under accession number PRJNA296814. The raw and processed metabolomics data is available at the NIH Common Fund's National Metabolomics Data Repository (NMDR) website, the Metabolomics Workbench, \url{https://www.metabolomicsworkbench.org} where it has been assigned Project ID PR001146. The data can be accessed directly via its project DOI: \url{https://doi.org/10.21228/M8K69N}. All intermediary analysis objects and scripts are available on GitHub. 

\section{Supplemental tables}
Table S1. Metabolites selected for targeted analysis and their potential biological functions.

Table S2: Primers used for bacterial 16S rRNA gene sequencing 

\section{Supplemental figures}

\begin{figure}
    \centering
    %\includegraphics{}
    \caption[Inter-omics Procrustes biplots comparing PCoA ordinations of untargeted metabolite profiles and taxonomic relative abundances for 6 weeks (left panels) (n = 158) and 12 months (right panels) (n = 262).]{Inter-omics Procrustes biplots comparing PCoA ordinations of untargeted metabolite profiles and taxonomic relative abundances for 6 weeks (left panels) (n = 158) and 12 months (right panels) (n = 262). Top panels present analyses based on ordinations from Euclidean distances of genus level abundances after centered log ratio transformation and Euclidean distances of arcsine square root transformed metabolite relative abundances. Bottom panel presents analyses based on generalized Unifrac distance of amplicon sequence variant (ASV) relative abundances and Euclidean distances of arcsine square root transformed metabolite relative abundances.}
    \label{fig:b1}
\end{figure}

\begin{figure}
    \centering
    %\includegraphics{}
    \caption[Pairwise Spearman correlation of metabolite bins and genus-level taxonomic abundances for 6-weeks (panel A, N = 158) and 12-months (panel B, N = 282) infants.]{Pairwise Spearman correlation of metabolite bins and genus-level taxonomic abundances for 6-weeks (panel A, N = 158) and 12-months (panel B, N = 282) infants. Left panel displays the overall correlation pattern, where non-significant correlations are not colored (false discovery rate (FDR) controlled q-value < 0.05). Right panel displays the same heatmap restricted to taxa and metabolites selected by the sparse CCA procedure. Additionally, correlation coefficient of the first sCCA variate pair, bootstrapped 95\% confidence interval and permutation p-value are also reported.}
    \label{fig:b2}
\end{figure}

\begin{figure}
    \centering
    %\includegraphics{}
    \caption[Comparative analysis predictive model performance across all metabolites in the untargeted dataset for both 6-weeks (n = 158) and 12-months (n = 282) timepoints.]{Comparative analysis predictive model performance across all metabolites in the untargeted dataset for both 6-weeks (n = 158) and 12-months (n = 282) timepoints. Top panel shows superimposed boxplots and violin plots of the distribution of predictive posterior mean for each evaluation metric across all 208 spectral bins. Bottom panels show aggregated model rankings for all metabolites using R-squared (left) and spearman correlation (right) using Borda scores (Methods).}
    \label{fig:b3}
\end{figure}


Figure S4. Results for positive (Panel A) and negative simulations (Panel B). Positive simulations were conducted based on bootstrapped resamples of the original data (12-month time point) and a normally distributed outcome vector which represented a log-transformed metabolite profile. Different levels of model saturation (horizontal, model sparsity (spar) at 0.05, 0.1, 0.5, 0.95) and effect sizes (vertical, signal-to-noise ratio (snr) at 0.5, 0.7, 3, 5) were assessed, with 100 data sets generated for each setting combination. Negative simulations were conducted based on permutations of the original data (12-month time point), with a total of 1000 permutations. Highly negative outliers were removed for the purposes of visualization
Figure S5. Inter-omics Procrustes biplots comparing PCoA ordinations of targeted metabolite profiles and taxonomic relative abundances in the sensitivity analyses for 6 weeks (left panels) (n = 65) and 12 months (right panels) (n = 65). Top panels present analyses based on ordinations from Euclidean distances of genus level abundances after centered log ratio transformation and Euclidean distances of arcsine square root transformed metabolite relative abundances. Bottom panel presents analyses based on generalized Unifrac distance of amplicon sequence variant (ASV) relative abundances and Euclidean distances of arcsine square root transformed metabolite relative abundances
Figure S6. Inter-omics Procrustes biplots comparing PCoA ordinations of untargeted metabolite bin relative concentrations and taxonomic relative abundances in the sensitivity analyses for 6 weeks (left panels) (n = 65) and 12 months (right panels) (n = 65). Top panels present analyses based on ordinations from Euclidean distances of genus level abundances after centered log ratio transformation and Euclidean distances of arcsine square root transformed metabolite relative abundances. Bottom panel presents analyses based on generalized Unifrac distance of amplicon sequence variant (ASV) relative abundances and Euclidean distances of arcsine square root transformed metabolite relative abundances
Figure S7. Pairwise spearman correlation of concentration-fitted targeted metabolite concentrations and genus-level taxonomic abundances for 6-weeks (panel A, N = 65) and 12-months (panel B, N = 65) infants in sensitivity analyses. Left panel displays the overall correlation pattern, where non-significant correlations are not colored (FDR controlled q-value < 0.05). Right panel displays the same heatmap restricted to taxa and metabolites selected by the sCCA procedure. Additionally, correlation coefficient of the first sCCA variate pair, bootstrapped 95\% confidence interval (nboot = 5000) and permutation p-value (nperm = 1000) are also reported.
Figure S8. Pairwise spearman correlation of untargeted metabolite bin relative concentrations and genus-level taxonomic abundances for 6-weeks (panel A, N = 65) and 12-months (panel B, N = 65) infants in sensitivity analyses. Left panel displays the overall correlation pattern, where non-significant correlations are not colored (FDR controlled q-value < 0.05). Right panel displays the same heatmap restricted to taxa and metabolites selected by the sCCA procedure. Additionally, correlation coefficient of the first sCCA variate pair, bootstrapped 95\% confidence interval (nboot = 5000) and permutation p-value (nperm = 1000) are also reported
Figure S9: Spearman correlation coefficients and 95\% confidence intervals of significant correlations (q-value < 0.05) between metabolite concentrations in the targeted data set and the abundances of pathways that produce them. Pathway abundances were obtained via PICRUSt2 predictions, with pathway-metabolite relationship retrieved from MetaCyc database. Both 6-week (n = 158) and 12-month (n = 282) samples are represented
Figure S10. Top five contributors at the Genus level for each significantly correlated pathway-metabolite pair obtained using observed metabolite concentrations and predicted pathway abundances (spearman correlation with q-value < 0.05). Panel A represents 6-week samples while panel B represents samples at 12-months. Relative contributions are calculated as the total number of copies of genes mapped to a pathway across all samples per Genus over the total number of gene copies assigned to that pathway. 
Figure S11. Heatmap representing overall spearman correlations between predicted pathway abundances (obtained via PICRUSt2) and metabolite concentrations in the targeted data set regardless of pathway-metabolite annotations. Both 6-week (n = 158) (Panel A) and 12-month (n = 282) (Panel B) samples are presented. Non-significant correlations (q-value > 0.05) are not colored. 
